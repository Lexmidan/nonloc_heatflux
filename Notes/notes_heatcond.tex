%\documentclass[10pt]{book}
\documentclass[12pt]{article}
%\documentclass[9pt]{article}
%\documentclass[11pt]{article}
%\documentclass[12pt]{report}

\usepackage[final]{graphicx}
\usepackage{amssymb,amsmath}
\usepackage{isomath}
\usepackage{bm}
\usepackage{color}
%\usepackage[vlined,ruled]{algorithm2e}

\textheight24.2cm
%\textwidth19.5cm
\textwidth18cm
\topmargin-2cm
\oddsidemargin-10mm
\evensidemargin-10mm
\parskip8pt
\parindent0pt
%\displayskip0pt

%\lineskip1.5em
\definecolor{dgreen}{rgb}{0.01,0.5,0.01}


%\newcommand{\func}[1]{\textcolor{blue}{\mathrm{\bf #1}}}
%\newcommand{\funct}[1]{\textcolor{blue}{{\bf #1}}}
\newcommand{\ffile}[1]{\textcolor{blue}{SRC/{#1}}}
\newcommand{\fproc}[2]{\textcolor{blue}{SRC/{#1}:}\textcolor{red}{\tt{#2}}}

\newcommand{\fdfile}[1]{\textcolor{blue}{SRC/{#1}}}
\newcommand{\fdprog}[1]{\textcolor{red}{{#1}}}
\newcommand{\fdsubr}[1]{\textcolor{red}{{#1}}}
\newcommand{\fdfunc}[1]{\textcolor{red}{{#1}}}

\newcommand{\fmdfile}[1]{\textcolor{blue}\mathrm{SRC/{#1}}}
\newcommand{\fmdprog}[1]{\textcolor{red}{\mathrm{#1}}}
\newcommand{\fmdsubr}[1]{\textcolor{red}{\mathrm{#1}}}
\newcommand{\fmdfunc}[1]{\textcolor{red}{\mathrm{#1}}}

\newcommand{\fcfile}[1]{\textcolor{blue}{file \fdfile{#1}}}
\newcommand{\fcprog}[1]{\textcolor{red}{program \fdprog{#1}}}
\newcommand{\fcsubr}[1]{\textcolor{red}{subroutine \fdsubr{#1}}}
\newcommand{\fcfunc}[1]{\textcolor{red}{function \fdfunc{#1}}}

\newcommand{\comc}[1]{\textcolor{dgreen}{\fbox{{#1}}}}
\newcommand{\comalg}[1]{\textcolor{dgreen}{$\blacktriangleright$ Alg.~\ref{#1}}}


\newcommand{\dx}{\Delta x}
\newcommand{\dy}{\Delta y}
\newcommand{\dt}{\Delta t}
\newcommand{\dftil}{\Delta \tilde{f}}

\newcommand{\oh}{\frac{1}{2}}

\newcommand{\im}{i-1}
\newcommand{\ip}{i+1}
\newcommand{\jm}{j-1}
\newcommand{\jp}{j+1}
\newcommand{\imh}{i-\oh}
\newcommand{\iph}{i+\oh}
\newcommand{\jmh}{j-\oh}
\newcommand{\jph}{j+\oh}

\newcommand{\imht}{i-1/2}
\newcommand{\ipht}{i+1/2}
\newcommand{\jmht}{j-1/2}
\newcommand{\jpht}{j+1/2}

\newcommand{\np}{n+1}
\newcommand{\nph}{n+\oh}
\newcommand{\npht}{n+1/2}

\def\cond{\kappa}
\def\clog{\lambda_{ei}}

\begin{document}

%\renewcommand[1]{\exp{#1}}{\mathrm{e^{#1}}}

%---------------------------------------------------------------
\begin{center}
\hrulefill\\[4mm]
{\Large\bf Calculation of Heat Conduction Utilizing Neural Networks}\\[4mm]
{\large Aleksandr Bogdanov,\quad
        Milan Holec,\quad
        Pavel V\'{a}chal}\\[3mm]
%{\large Richard Liska,\quad
%        Milan Kucha\v{r}\'{i}k,\quad
%        Pavel V\'{a}chal}\\[3mm]
{\normalsize (and whoever will work on this)}\\[8mm]
{\it Notes compiled by PV} {\tt(pavel.vachal@fjfi.cvut.cz)}\\[3mm]
{\it \today}\\[2mm]
\hrulefill
\end{center}
%---------------------------------------------------------------
%---------------------------------------------------------------
\tableofcontents

\hrulefill
%\newpage
%---------------------------------------------------------------
%---------------------------------------------------------------
\section{Governing Equations}

We want to solve
\begin{align}\label{pde}
  \frac{d\,\varepsilon(x)}{dt} &= -\vec{\nabla} \cdot \vec{q}(x), 
\end{align}
where
\begin{align}\label{q_TbNT}
    \vec{q}(x)&= - \alpha(x) \cond(x) T(x)^{\beta(x)} \vec{\nabla} T(x),  
\end{align}
and
\begin{align}\label{e_kB}
  \varepsilon(x)
  &= C_V(x)\,T(x) = \frac{3}{2}n(x)\,k_B\,T(x) .
\end{align}
The~Gauss units are as follwoing 
$\varepsilon\left[\frac{\mathrm{erg}}{\mathrm{cm}^3}\right], T [\mathrm{eV}], q [\frac{\mathrm{erg}}{\mathrm{cm}^2}]$
with $k_B= 1.380649\times 10^{-16} \frac{\mathrm{erg}}{\mathrm{K}}
      = 1.602178\times 10^{-12} \frac{\mathrm{erg}}{\mathrm{eV}}$ and $\alpha, \beta$ are unitless.
The~conductivity $\cond$ in our model is defined in Gauss units as
\begin{align}
  \cond(x)&= \frac{Z(x) + 0.24}{Z(x) + 4.2}~ 
    \frac{1.31 \times 10^{10}}{Z(x) \clog(x)}~
    \tau^{\beta(x)-\frac{5}{2}} , 
\end{align}
where Coulomb logarithm 
$\clog(x) = 23 - \ln\left(\frac{\sqrt{n(x)} Z(x)}{T(x)^{3/2}}\right)$ 
(see Plasma Formularly) and $\tau = \frac{1}{T_\mathrm{preheat}}$. 
{\bf We use $T_\mathrm{preheat} = 1000~\mathrm{eV}$ to safely include 
the~preheat region} of the~hohlraum wall simulation. Note that 
$1\,\mathrm{erg}= 1\,\frac{\mathrm{g}\cdot\mathrm{cm}^2}{\mathrm{s}^2}
= 10^{-7}\,\mathrm{J}$, which might be used to convert heat flux for 
benchmarking.

In order to solve \eqref{pde}, \eqref{q_TbNT}, and \eqref{e_kB}, we define 
\begin{align}\label{F}
  F&:= C_V\,\frac{d\,T}{dt}
  - \frac{d}{dx}\underset{q}{\underbrace{%
      \left(\alpha \cond T^\beta \frac{d}{dx}T\right)}}
%      ~\overset{\boldsymbol{!}}{=}~0.
\end{align}
and our task will be to find $T$ that solves
\begin{align}\label{Feq}
  F(T)&=0.
\end{align}


%---------------------------------------------------------------
\section{Discretization}

\begin{figure}[htb]%[tb]%
\centerline{\includegraphics[width=0.8\textwidth]{pics/mesh-stag_withBC}}
\caption{\it Discretization of the 1D problem}
\label{fig:discr}
\end{figure}


We discretize the problem on a general 1D mesh with cell-related variables indexed
by integers and node-related ones by half-integers, as shown in Fig.~\ref{fig:discr}.

Thus the cell volume is
\begin{align}
\dx_i&=x_{\iph}-x_{\imh},
\end{align}
while the volume of the node-assigned dual cell 
(distance between neighboring cell centers) is
\begin{align}
\dx_{\iph}&=x_{i+1}-x_{i} = \frac{\dx_i+\dx_{\ip}}{2}.
\end{align}
The divergence of $q$ in \eqref{pde} is discretized on the primary cell by 
the finite difference
\begin{align}
  \left.\vec{\nabla} \cdot \vec{q}\right\rvert_i \overset{1D}{=}
  \left.\frac{d\,q}{dx}\right\rvert_i
  &\approx  \frac{q_{\iph}-q_{\imh}}{\dx_i} ,
\end{align}
with the nodal value of the~flux \eqref{q_TbNT} being approximated as
\begin{align}
  q_{\iph} &= \overline{\left(\alpha\cond T^\beta\right)}_{\iph}\,
  \frac{T_{i+1}-T_i}{\dx_{\iph}},
\end{align}
where\;
$\overline{\left(\alpha\cond T^\beta\right)}_{\iph}$\;
is obtained by some kind of averaging from the two connected cells, for example
\begin{subequations}\label{ak_aver}
\begin{align}
  \overline{\left(\alpha\cond T^\beta\right)}_{\iph}
  &= \frac{\left(\alpha\cond T^\beta\right)_i
         + \left(\alpha\cond T^\beta\right)_{i+1}}{2},\\
  \overline{\left(\alpha\cond T^\beta\right)}_{\iph}
  &= \frac{\dx_i\,\left(\alpha\cond T^\beta\right)_i
         + \dx_{i+1}\,\left(\alpha\cond T^\beta\right)_{i+1}}
           {\dx_i+\dx_{i+1}}, \quad\text{or}\\
  \overline{\left(\alpha\cond T^\beta\right)}_{\iph}
  &= \frac{\frac{1}{\dx_i}\,\left(\alpha\cond T^\beta\right)_i
         + \frac{1}{\dx_{i+1}}\,\left(\alpha\cond T^\beta\right)_{i+1}}
           {\frac{1}{\dx_i}+\frac{1}{\dx_{i+1}}},
\end{align}
\end{subequations}
where we denoted
\begin{align}
  \left(\alpha\cond T^\beta\right)_{j}
  &=\alpha_j \cond_j T_j^{\beta_j}.
\end{align}

Discretizing \eqref{Feq}, resp. \eqref{pde}, \eqref{q_TbNT}, and \eqref{e_kB}
over the $i$-th cell, we have
\begin{align}\label{F_disc}
  F_i&:= C_{Vi}\,\frac{d\,T_i}{dt}
  - \frac{\overline{\left(\alpha\cond T^\beta\right)}_{\iph}
          \frac{T_{i+1}-T_{i}}{\dx_{\iph}}
        - \overline{\left(\alpha\cond T^\beta\right)}_{\imh}
          \frac{T_{i}-T_{i-1}}{\dx_{\imh}}
  }{\dx_i}
\end{align}
with space-dependent $\alpha$ and $\beta$ being provided by the neural network
and $k$, $C_V$ being also functions of $x$:
\begin{align}
 \alpha&=\alpha(\mathrm{NN}(x)),&
 \beta&=\beta(\mathrm{NN}(x)),&
 k&=k(Z(x)),&
 C_V&=C_V(n(x)).
\end{align}
At this point let us remark, that classical heat conductivity in plasma
uses constant \,$\beta=5/2$,\, which further simplifies the equations. This is the case
for example in [Silar, vyzkumak, 2009]. However, there the problem is transformed
using \,$\theta=T^{7/2}$\, and solved by a mimetic scheme, whereas here we are going
to proceed by Newton's iterative method.

For a regular mesh (i.e., with equidistant nodes), we have
\begin{align}
  \dx &= \dx_i = \dx_{\iph}, \; \forall i,
\end{align}
and thus \eqref{F_disc} simplifies to
\begin{align}\label{F_disc_eq}
  F_i&= C_{Vi}\,\frac{d\,T_i}{dt}
  - \frac{1}{\dx^2}\left\lgroup
          \overline{\left(\alpha\cond T^\beta\right)}_{\iph}
          \left(T_{i+1}-T_{i}\right)
        - \overline{\left(\alpha\cond T^\beta\right)}_{\imh}
          \left(T_{i}-T_{i-1}\right)
          \right\rgroup
\end{align}
and all three types of averaging \eqref{ak_aver} are equivalent:
\begin{align}\label{ak_aver_eq}
  \overline{\left(\alpha\cond T^\beta\right)}_{\iph}
  &= \frac{\left(\alpha\cond T^\beta\right)_i
         + \left(\alpha\cond T^\beta\right)_{i+1}}{2}.
\end{align}

There are several ways to solve \eqref{Feq}, that is,
in the discrete case
\begin{align}\label{Feq_disc}
F_i(\vectorsym{T})&=0,\;\forall i.
\end{align}

Replacing also the time derivative by a finite difference,
\eqref{F_disc_eq} becomes
\begin{align}\label{eq:Fprecise}
  F_i(\vectorsym{T}) &= C_{V_i}\,\frac{T_i-T_i^{[t-\dt]}}{\dt}
  - \frac{1}{\dx^2}\left\lgroup
          {\textstyle{\overline{\left(\alpha\cond T^\beta\right)}}}_{\iph}
          \left(T_{i+1}-T_{i}\right)
        - {\textstyle{\overline{\left(\alpha\cond T^\beta\right)}}}_{\imh}
          \left(T_{i}-T_{i-1}\right)
    \right\rgroup,
\end{align}
where $T_i^{[t-\dt]}$ is the temperature at the previous time level $t-\dt$.
Note that by using temperature at the actual time level $t$
in the spatial difference (the term in parentheses),
we are aiming at implicit schemes, so that the time step $\dt$ is not
overrestricted by stability requirements.


%---------------------------------------------------------------
\section{Solving the Discrete Problem}
%---------------------------------------------------------------
\subsection{Newton's Iteration}

For simplicity, let's take in each equation the value of \,$(\alpha\,k/\beta)$\,
from the actual cell instead of using nodal averages at its endpoints.
Then we have a system similar
to \eqref{Feq_disc} with the $i$-th equation being
\begin{align}\label{F_disc_eq_cellb}
  F^*_i(\vectorsym{T})&= C_{Vi}\,\frac{T_i-T_i^{[t-\dt]}}{\dt}
  - \alpha_i \cond_i T_i^{\beta_i} \,
          \frac{T_{i-1} - 2\,T_{i} + T_{i+1}}{\dx^2}.
\end{align}
The Jacobian of such system is a tridiagonal matrix with the elements
\begin{subequations}\label{dF_disc_eq_cellb}
  \begin{align}
    J_{i,i}
    &= \frac{\partial F^*_i}{\partial\,T_i}
    =  \frac{C_{Vi}}{\dt}    
        + 2\,\frac{\alpha_i\,k_i}{\dx^2}(\beta_i+1)\,T_{i}^{\beta_i}
    \\
    J_{i,i\pm 1}
    &= \frac{\partial F^*_i}{\partial\,T_{i\pm 1}}
    = -\,\frac{\alpha_i \cond_i T_i^{\beta_i}}{\dx^2}.
\end{align}
\end{subequations}

Using precise definition of nonlinear functional $F$ given by 
\eqref{eq:Fprecise} and approximate Jacobian \eqref{dF_disc_eq_cellb},
we can now perform the $k$-th iteration of Newton's method:
\begin{align}
  \vectorsym{T}^{(k+1)} = \vectorsym{T}^{(k)} -
                        \tensorsym{J}^{-1}\, \vectorsym{F}(\vectorsym{T}^{(k)}).
\end{align}

Keep in mind, that the superscript $^{(k)}$ stands for Newton's iteration,
not for the evolution in time! Therefore, $T_i^{[t-\dt]}$ in \eqref{F_disc_eq_cellb}
stays the same in all
iterations at given time $t$, until the solution $\vectorsym{T}$ at this time 
level has converged.


If we require an exact Jacobian, we can obtain it by deriving from the equation \eqref{eq:Fprecise} using \eqref{ak_aver}. Consequently, it can be deduced that:

\begin{subequations}\label{Jexact}
  \begin{align}
    J_{i,i}
    &= \frac{\partial F_i}{\partial\,T_i}
    =  \frac{C_{Vi}}{\dt}    
        + \frac{1}{2}\alpha_{i+1} k_{i+1}T^{\beta_{i+1}}_{i+1}
        + \frac{1}{2}\alpha_{i-1} k_{i-1}T^{\beta_{i-1}}_{i-1}
        + (\beta_i+1)\alpha_i k_i T^{\beta_i}_i
        - \beta_i \alpha_i k_i T^{\beta_i-1}_i \left(T_{i+1}+T_{i+1}\right)
    \\
    J_{i,i\pm 1} 
    &= \frac{\partial F^*_i}{\partial\,T_{i\pm 1}}
    = \frac{1}{2} \beta_{i\pm 1} \alpha_{i\pm 1} k_{i\pm 1} T_{i\pm 1}^{\beta_{i\pm 1}-1} T_i -
        \frac{1}{2} \left( \beta_{i\pm 1}+1\right)\alpha_{i\pm 1} k_{i\pm 1} T_{i\pm 1}^{\beta_{i\pm 1}}  -\frac{1}{2} \alpha_i k_i T_{i}^{\beta_i}
\end{align}
\end{subequations}
%Note that in [Silar, vyzkumak, 2009], a similar process is applied to the
%nonlinear case with non-zero but constant $\beta=5/2$, where the system is
%first transformed into the variable $\theta=T^{7/2}$.

%---------------------------------------------------------------
\subsection{Boundary Conditions}

We will enforce 
\begin{align}
  \nabla T &= 0
\end{align}
on both ends of the domain. To do this on a mesh of $N$ cells indexed from
$1$ to $N$
(see Fig.~\ref{fig:discr}), we formally introduce the ghost values
\begin{align}
  T_0 &= T_1,&   T_{N+1}&=T_N.
\end{align}
Inserting them into the equations \eqref{F_disc_eq_cellb}
for the boundary cells ($i=1$ resp. $i=N$), we get
\begin{subequations}
  \begin{align}
    %\label{F_disc_eq_cellb}
    F_1(\vectorsym{T}) &:= C_{V_1}\,\frac{T_1-T_1^{[t-\dt]}}{\dt}
     - \frac{
       \textstyle{\overline{\left(\alpha\cond T^\beta\right)}}_{\frac{3}{2}}
        ( T_{2}-T_{1}) }{\dx^2} ,
    \\
    F_N(\vectorsym{T}) &:= C_{V_N}\,\frac{T_N-T_N^{[t-\dt]}}{\dt}
     + \frac{
       \textstyle{\overline{\left(\alpha\cond T^\beta\right)}}_{N - \frac{1}{2}}
        ( T_{N}-T_{N-1}) }{\dx^2} ,
\end{align}
\end{subequations}
which immediately yields the first and last row of the Jacobian matrix
\begin{subequations}
  \begin{align}
    J_{1,1}
    &~=~ \frac{\partial F^*_1}{\partial\,T_1}
     ~=~ \frac{C_{V_1}}{\dt}    
        + \frac{\alpha_1 \cond_1}{\dx^2} (\beta_1+1)\,T_{1}^{\beta_1},\\
    J_{1,2}
    &~=~\frac{\partial F^*_1}{\partial\,T_2}
     ~=~ -\,\frac{\alpha_1 \cond_1 T_1^{\beta_1}}{\dx^2},\\
    J_{N,N-1}
    &~=~\frac{\partial F^*_N}{\partial\,T_{N-1}}
     ~=~ -\,\frac{\alpha_N \cond_N T_{N}^{\beta_N}}{\dx^2},\\
    J_{N,N}
    &~=~\frac{\partial F^*_N}{\partial\,T_N}
     ~=~ \frac{C_{V_N}}{\dt}    
        + \frac{\alpha_N \cond_N}{\dx^2} (\beta_N+1)\,T_{N}^{\beta_N}.
\end{align}
\end{subequations}


%---------------------------------------------------------------
\section{The Linear Case}

Let us now look at the special case of linear heat conduction,
that is,
\begin{align}
  \beta_i&\equiv 0\;,\forall i
\end{align}
and for simplicity let us also use a constant coefficient
\begin{align}
 a&=\alpha_i\,k_i,\,\forall i
\end{align}
and constant heat capacity.
In other words, we are solving
\begin{align}
  C_V\frac{d\,T}{dt} &= a\,\frac{d^2\,T}{dx^2},
\end{align}
which we discretize (again aiming at the implicit scheme),
define the functional
\begin{align}\label{F_const}
  F_i&=C_V\frac{T_i-T_i^{[t-\dt]}}{\dt} - a\,\frac{T_{i-1}-2\,T_i+T_{i+1}}{\dx^2}
\end{align}
and calculate
\begin{align}\label{eq_const}
  F_i&=0,\; \forall i.
\end{align}
If Newton's iteration was to be used, the Jacobian would be a constant tridiagonal
matrix with elements
\begin{subequations}
  \begin{align}
    J_{i,i}
    &= \frac{\partial F_i}{\partial\,T_i}
    =  \frac{C_V}{\dt} + 2\,\frac{a}{\dx^2}\\
    J_{i,i\pm 1}
    &= \frac{\partial F_i}{\partial\,T_{i\pm 1}}
    = -\,\frac{a}{\dx^2}.
\end{align}
\end{subequations}
However, for such a simple problem it is not necessary to iterate
or even explicitly formulate the Jacobian.
We can simply rewrite (basically just reorder) \eqref{eq_const} with \eqref{F_const} as
\begin{align}
  T_i - \frac{a}{C_V}\,\frac{\dt}{\dx^2}\left(T_{i-1}-2\,T_i+T_{i+1}\right) &= T_i^{[t-\dt]},
\end{align}
that is,
\begin{align}
  - \frac{a}{C_V}\,\frac{\dt}{\dx^2}\,T_{i-1}
  + \left(1+2\frac{a}{C_V}\,\frac{\dt}{\dx^2}\right)T_i
  - \frac{a}{C_V}\,\frac{\dt}{\dx^2}\,T_{i+1}
   &= T_i^{[t-\dt]},
\end{align}
and denoting
\begin{align}
  b &= \frac{a}{C_V}
\end{align}
we recover 
\begin{align}
  - b\,\frac{\dt}{\dx^2}\,T_{i-1}
  + \left(1+2\,b\,\frac{\dt}{\dx^2}\right)T_i
  - b\,\frac{\dt}{\dx^2}\,T_{i+1}
   &= T_i^{[t-\dt]},
\end{align}
which yields a simple linear system
\begin{align}
   \tensorsym{A}\;\vectorsym{T}&=\vectorsym{T}^{[t-\dt]},
\end{align}
with a tridiagonal constant matrix $\tensorsym{A}$
as it can be found in many textbooks on implicit methods for parabolic problems.

%---------------------------------------------------------------
%---------------------------------------------------------------
\end{document}

%---------------------------------------------------------------
%---------------------------------------------------------------
%---------------------------------------------------------------
%---------------------------------------------------------------


%---------------------------------------------------------------
\section{Finite Differences}

Equation \eqref{Feq} can be also solved using finite differences.
To demonstrate the process, assume the
(not very realistic)
special case which avoids
the explicit nonlinearity, that is, $\beta\equiv 0$, simplifying 
the discrete functional \eqref{F_disc_eq}  to
\begin{align}\label{F_disc_eq_b0}
  F_i&:= \frac{3}{2}n_i\,k_B\,\frac{d\,T_i}{dt}
  - \frac{1}{\dx^2}\left\lgroup
          \overline{\left(\alpha\,k\right)}_{\iph}
          \left(T_{i+1}-T_{i}\right)
        - \overline{\left(\alpha\,k\right)}_{\imh}
          \left(T_{i}-T_{i-1}\right)
    \right\rgroup.
\end{align}
Now, setting the functional equal to $0$, replacing the time derivative by
finite difference between time levels $t^n$ and $t^{n+1}$ and using the implicit scheme
(that is, using the temperatures at the right-hand side at time $t^{n+1}$), we have
\begin{align}\label{F_disc_eq_b0_1}
  0 &= \frac{3}{2}n_i\,k_B\,\frac{T_i^{n+1}-T_i^n}{\dt}
  - \frac{1}{\dx^2}\left\lgroup
          \overline{\left(\alpha\,k\right)}_{\iph}
          \left(T_{i+1}^{n+1}-T_{i}^{n+1}\right)
        - \overline{\left(\alpha\,k\right)}_{\imh}
          \left(T_{i}^{n+1}-T_{i-1}^{n+1}\right)
    \right\rgroup.
\end{align}
Let us repeat that due to the special choice $\beta\equiv 0$, the system is linear in $T$
and the superscripts over $T$ now denote the time level, not a power of $T$.
After some reordering, we have
\begin{align}\label{F_disc_eq_b0_2}
  T_i^{n+1}
  &= T_i^n + \frac{2}{3\,n_i\,k_B}\frac{\dt}{\dx^2}
    \left\lgroup
          \overline{\left(\alpha\,k\right)}_{\imh}\,T_{i-1}^{n+1}
          - \left(\overline{\left(\alpha\,k\right)}_{\imh}
                 +\overline{\left(\alpha\,k\right)}_{\iph}\right)T_{i}^{n+1}
          +\overline{\left(\alpha\,k\right)}_{\iph}
           T_{i+1}^{n+1}
    \right\rgroup,
\end{align}
which can be also written as
\begin{align}
\nonumber
  - \frac{2}{3\,n_i\,k_B}\frac{\dt}{\dx^2}
     \overline{\left(\alpha\,k\right)}_{\imh}\,T_{i-1}^{n+1}
   + \left(1+\frac{2}{3\,n_i\,k_B}\frac{\dt}{\dx^2}
    \left(\overline{\left(\alpha\,k\right)}_{\imh}
         +\overline{\left(\alpha\,k\right)}_{\iph}\right)\right)T_{i}^{n+1}&\\
   - \frac{2}{3\,n_i\,k_B}\frac{\dt}{\dx^2}
    \overline{\left(\alpha\,k\right)}_{\iph}\,T_{i+1}^{n+1}
    &= T_i^n,       
    \label{F_disc_eq_b0_3}
\end{align}
which provides the $i$-th line (for an interior cell) of a linear system
\begin{align}
   \mathbf{A}\,\mathbf{T}^{n+1}&=\mathbf{T}^{n}
\end{align}
with a tridiagonal matrix $\mathbf{A}$.

Note that in [Silar, vyzkumak, 2009], a similar process is applied to the
nonlinear case with non-zero but constant $\beta=5/2$, where the system is
first transformed into the variable $\theta=T^{7/2}$.

Finally, let us remark that in the idealized case of constant heat conduction
coefficient
\begin{align}
  b &~=~ \frac{2}{3\,n_i\,k_B}\overline{\left(\alpha\,k\right)}_{\imh}
     ~=~ \frac{2}{3\,n_i\,k_B}\overline{\left(\alpha\,k\right)}_{\iph},\qquad \forall i,
\end{align}
we recover the classical tridiagonal system with the line (for an interior cell)
of the form
\begin{align}
\nonumber
  - b\,\frac{\dt}{\dx^2}\,T_{i-1}^{n+1}
   + \left(1+2\,b\right)T_{i}^{n+1}
   - b\,\frac{\dt}{\dx^2}\,T_{i+1}^{n+1}
    &~=~ T_i^n,       
%    \label{F_disc_eq_b0_3}
\end{align}
which can be found in many textbooks on implicit methods for parabolic problems.

%- - - - - - - - - - - - - - - - - - - - - - - - - - - - - - - -
%\subsubsection{}


%---------------------------------------------------------------
%---------------------------------------------------------------
%\begin{verbatim}
%\end{verbatim}
%---------------------------------------------------------------
%---------------------------------------------------------------
%\section{}
%---------------------------------------------------------------
%---------------------------------------------------------------
\end{document}
